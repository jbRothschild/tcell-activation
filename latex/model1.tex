\subsection{\label{sec:net1}Network 1}

We will focus first on a simple network which looks something like this [STILL NEEDS TO BE FINISHED]

\begin{figure}[h]
\tikzstyle{species}=[circle,,minimum size=0.75cm, line width=1pt,draw=black!]
\tikzstyle{square}=[rectangle,thick,minimum size=0.75cm,draw=blue!80,fill=blue!20]
\tikzstyle{vspecies}=[rectangle, minimum size=0.75cm,draw=blue!80,fill=blue!20]
\tikzstyle{square}=[rectangle,thick,minimum size=0.75cm,draw=red!80,fill=blue!20]
\tikzstyle{fspecies}=[rectangle, minimum size=0.5cm,draw=red!80,fill=red!20]
\begin{tikzpicture}[auto, outer sep=3pt, node distance=2.5cm,>=latex']
\node [species] (ag) {ag};
\node [species, above right of = ag] (Tac) {$T_{act}$};
\node [species,below right of = ag] (T) {$T_{pr}$};
\node [species,below right of = Tac] (C) {$C_{IL2}$};
\node [species,right of = C] (Treg) {$T_{reg}$};
%\draw [->,thick] (ag) --  node {$V_1$} (Tac) ;
\draw [->,thick] (ag) -- (Tac);
\draw [->,thick] (Tac) -- (T);
\draw [->,thick] (Tac) -- (C);
\draw [->,thick] (C) -- (Treg);
\draw [-|,thick] (Treg) -- (C);
\draw [-|,thick] (T) -- (ag);

%\draw [-|,thick] (Tac.0) arc (-90:160:4mm);

\draw [-|,thick] (Tac) to [out=0,in=60,loop,looseness=4] (Tac);
\end{tikzpicture}
\end{figure}

\subsubsection{\label{sec:eq1}The differential equations}

\begin{equation}
\begin{aligned}
\frac{d ag}{dt} & = -0.1 ag(t) T_{pr}(t)\\
\frac{d T{ac}}{dt} & = 0.2 ag(t) ( 100 - T_{ac}(t) ) - T_{ac}(t)\\
\frac{d C_{IL2}}{dt} & = T_{ac} - 10\frac{ C_{IL2}(t)  T_{reg}(t) }{ 1 + C_{IL2}(t) }\\
\frac{d T_{reg}}{dt} & = \frac{ C_{IL2}(t) }{1 + C_{IL2}(t)} - 0.1 T_{reg}(t)\\
\frac{d T_{pr}}{dt} & = T_{pr} ( 1-\frac{1.3}{1+C_{IL2}(t)} ) + 0.1 T_{ac}
\end{aligned}
\label{me1}
\end{equation}

\subsubsection{\label{sec:res1}Results}

First, I investigated what the solution of these equations would look like for different initial concentrations of antigen.
\begin{widetext}

\begin{figure}
\begin{subfigure}{.4\textwidth}
  \centering
  \includegraphics[width=1.\linewidth]{Network_ag0_1.png}
  \caption{$ag(0)=0.1$}
  \label{fig:sfig1}
\end{subfigure}%
\begin{subfigure}{.4\textwidth}
  \centering
  \includegraphics[width=1.\linewidth]{Network_ag0_3.png}
  \caption{$ag(0)=0.3$}
  \label{fig:sfig2}
\end{subfigure}
\begin{subfigure}{.4\textwidth}
  \centering
  \includegraphics[width=1.\linewidth]{Network_ag100.png}
  \caption{$ag(0)=100$}
  \label{fig:sfig3}
\end{subfigure}%
\begin{subfigure}{.4\textwidth}
  \centering
  \includegraphics[width=1.\linewidth]{Network_ag200.png}
  \caption{$ag(0)=200$}
  \label{fig:sfig4}
\end{subfigure}

\caption{Plots of solutions to network 1 with $ag(0)$ varied and all other initial concentrations set to 0.}
\label{fig:fig}
\end{figure}
\end{widetext}

According to these plots, there is always a slight activation of proliferating T cells (in blue). Is this biological? Are proliferating T cells always somewhat activated or is there a threshold below which no activation happens? If there is a strict threshold, there needs to be a harder cutoff in equations \ref{me1}, such as a hill function $\frac{x^n}{c^n+x^n}$ with n large ($\approx 10$).

In Figure \ref{vary_init_ag}, we clearly see that the maximum concentration of T cell is attainedat later times when the initial concentration of antigen is increased.

\begin{figure}[ht!]
	\centering
  	\includegraphics[width=1.\linewidth]{T_cell_by_initial_ag.png}
  	\caption{Change in proliferating T cell concentration}
  	\label{vary_init_ag}
\end{figure}

However we do not see that the increase in time is significant over a large range of initial concentrations of antigen. 
