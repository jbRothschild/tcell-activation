\begin{quotation}
Right now there is a model of coupled differential equations that explain the basic interactions. I should look at the how the starting concentration of antigen changes the dynamics of the T cell proliferation. What is the limit of the antigen on the explosion of T cell? In the same line of thought, how does the initial population of potentially activated T cells affect the number of proliferated T cells? Since activated T cells are a function of antigen and affinity, what are different limits?

There should be some `gobbling' up of the cytokine by the proliferating T cells. This is not in the model.

\textit{Further down the line...} We should look into the internal accumulation of Myk by T cells. When T cells divide there is a decrease in concentration fo Myk (which helps them divide). At some point the concentration is too low so there are no more divisions.

Need to read some papers by Gregoire...
Additionally are there other mechanisms that stop antigen presentation?
\end{quotation}
